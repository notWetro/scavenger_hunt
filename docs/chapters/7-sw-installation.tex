\chapter{Inbetriebnahme} \label{cha:inbetriebnahme}

\section{Einführung}

In diesem Kapitel werden die Maßnahmen für das Deployment der Anwendung näher beschrieben. Anhand einer Übersicht wird gezeigt, wie das System im laufenden Zustand aussieht und wie die unterschiedlichen Anwendungen mit dem Backend interagieren.

\section{Backend Containerisierung}

\begin{figure}[H]
  \centering
  \includegraphics[width=\textwidth]{images/PrAr_Depl_Docker.png}
  \caption{UML-Diagramm für das Deployment}
  \label{fig:deployment:docker}
\end{figure}

Abbildung \ref{fig:deployment:docker} stellt die Deployment-Ansicht des Backends dar. Das Deployment wird durch die Docker Containerisierung ermöglicht (vgl. Kapitel \ref{cha:grundlagen:swtech:docker}). Die Konfiguration der verschiedenen Dienste erfolgt über \textit{Docker-Compose} definiert.

Der Proxy (vgl. Abbildung \ref{fig:deployment:docker} - blau) fungiert als zentraler Einstiegspunkt für die Anwendungen von außen. Er leitet alle Anfragen an den jeweils zuständigen Dienst weiter. Im Falle von falschen oder ungültigen Anfragen wird die Verbindungsanfrage beendet. Die Implementierung und Konfiguration des Proxies ist im Anhang \ref{appendix:code:dockercompose} beschrieben.

Die beiden Container \textit{Participants-Api} und \textit{Hunts-Api} (vgl. Abbildung \ref{fig:deployment:docker} - grün und violett) repräsentieren die deployten Release-Builds der entworfenen Services aus Kapitel \ref{cha:swentwurf:backend}. Diese verfügen jeweils über einen eigenen Datenbank-Container zur persistenten Datenhaltung.

Für die initiale Erstellung und Aktualisierung bestehender Datenbank-Container sind zusätzlich die beiden Migration-Container \textit{Participants-Migrations} und \textit{Hunts-Migrations} vorgesehen (vgl. Abbildung \ref{fig:deployment:docker} - grün und violett). Beide öffnen eine Verbindung zur jeweiligen Datenbank und erstellen die initial benötigten Datenbanktabellen, damit diese für die Verwendung im jeweiligen Service mit den aktuellen Datenbanktabellen existieren.

Die jeweiligen Dockerfiles sind im Anhang näher beschrieben (vgl. Anhang \ref{appendix:code:dockercompose}).

\section{Frontend Containerisierung}

Da das Frontend mit Web-Technologien (vgl. Kapitel \ref{cha:grundlagen:swtech:svelte}) implementiert wurde, kann es als plattformunabhängige Browseranwendung auf einem Webserver deployt werden. Dies ist auch über Docker Containerisierung möglich, wurde aber im Rahmen des Projektes nicht entworfen und getestet.

Für das Deployment der Hunt-Editor Web-App für Desktops wäre es auch möglich, eine echte Anwendung daraus zu machen. Hierfür stehen bereits implementierte Technologien wie \textit{Tauri} zur Verfügung (siehe \autocite{github:tauri} und \autocite{tauri:tauri}).
