\section{Technologien}

Für die Implementierung wurden zahlreiche Technologien der Software-Entwicklung in Betracht gezogen. In diesem Abschnitt werden die für das Projekt relevanten Frameworks und Plattformen beschrieben.

\subsection{ASP.NET}

ASP.NET ist ein Open-Source-Web-Framework, das von Microsoft entwickelt wurde und die Erstellung moderner, skalierbarer und leistungsfähiger Webanwendungen ermöglicht. Zudem werden in ASP.NET moderne Webtechnologien und -standards unterstützt, wodurch sich die Entwicklung von interaktiven und responsiven Webanwendungen erleichtert. Dies schließt auch Apis und Echtzeit-Kommunikation ein, die für die Anwendung relevant sein könnten. Diese Merkmale ermöglichen es auch, die Anwendung modular zu ergänzen, indem externe Apis wie OpenStreetMaps oder andere Dienste einfach angebunden werden können. \autocite{MicrosoftLearn:AspNet}

\subsubsection{Entity Framework Core}

\ac{efcore} ist ein leichtgewichtiges, erweiterbares und Open-Source-\ac{orm} Framework für .NET, das entwickelt wurde, um den Datenzugriff und die Datenmanipulation in .NET-Anwendungen zu vereinfachen. \ac{efcore} bietet Datenbankunabhängigkeit, da es verschiedene Datenbanksysteme wie SQL Server, MySQL, PostgreSQL und SQLite unterstützt.

Ein weiterer Vorteil von \ac{efcore} ist Möglichkeit, das Datenbankschema aus konkreten Models über Source-Code zu generieren, ohne \textit{Create-Table} SQL-Statements schreiben zu müssen. \ac{efcore} bietet hierbei eine einfach zu verstehende \textit{Fluent-Api} an. Zudem ermöglicht \ac{efcore} die einfache Verwaltung von Datenbankmigrationen, was es erlaubt, Änderungen am Datenmodell nachzuverfolgen und auf die Datenbank anzuwenden. Dies erleichtert die Wartung und Weiterentwicklung erheblich.

Ein zusätzliches Merkmal von \ac{efcore} ist die Integration von \ac{linq}, die es ermöglicht, komplexe Abfragen auf eine intuitive und typsichere Weise zu schreiben. Dies verbessert die Lesbarkeit und Wartbarkeit des Codes erheblich. Schließlich bietet \ac{efcore} verschiedene Mechanismen zur Performanceoptimierung, wie zum Beispiel asynchrone Abfragen und Caching-Strategien. \autocite{MicrosoftLearn:EfCore}

\subsection{Unity und Mobile-Apps}

Unity ist eine Plattform zur Entwicklung und Darstellung interaktiver 3D-Inhalte, die in vielen Bereichen wie Spieleentwicklung, Filmproduktion, Architekturvisualisierung und \ac{vr} eingesetzt wird. Sie ist eine der am weitesten verbreiteten Spiel-Engines weltweit und bietet eine Vielzahl an Werkzeugen und Funktionen, für das implementieren immersiver und realistischer Umgebungen.

Unity basiert auf einer Engine, die eine umfassende Sammlung von Software-Bibliotheken bereitstellt. Diese Bibliotheken ermöglichen die Verarbeitung von Grafiken, Physik, Sound und Benutzereingaben. Der Kern der Engine ist für die Rendern von 2D- und 3D-Grafiken verantwortlich, die in Echtzeit berechnet und auf dem Bildschirm angezeigt werden. Unity unterstützt die Programmiersprachen C\# und UnityScript (eine Art JavaScript, welche allerdings nicht mehr weiterentwickelt wird), wobei C\#  die primäre Sprache für die Entwicklung in Unity ist. \autocite{UnityDocs2024}

\subsubsection{AR-Foundation}

Für das Erstellen von Augmented-Reality Anwendungen bietet Unity unter der \textit{AR Foundation} grundlegende Bausteine an. Über vorgefertigte Projekt-Templates lässt sich eine funktionierende und ausführbare Demo-Anwendung für Android-Geräte generieren. Das AR Foundation Framework ermöglicht das Einbinden von Features wie beispielsweise Oberflächen-Erkennung, Objekt-, Bild- und Gesicht-Verfolgung sowie Sitzungs- und Geräte-Management. \autocite{Unity2024}

\subsubsection{Nützliche Bibliotheken}

\textit{Newtonsoft.Json.NET} ist eine in der Praxis verwendete Bibliothek in C\# bzw. .NET Anwendungen für das Serialisieren und Deserialisieren von Objekten im JSON-Format. Für die Verwendung in Unity-Anwendungen existiert ein GitHub-Fork (\textit{Json.Net.Unity3D}) und ermöglicht das Einbinden über eine \lstinline{.unitypackage} Datei. \autocite{SaladLab2024}

\textit{AR+GPS Location} ist eine Utility aus dem Unity-Asset-Store und ermöglicht eine Übersetzung von GPS-Koordinaten im Unity-Raum. Hierdurch können Objekte über die Angabe eines realen GPS-Standorts im Unity-Raum plaziert werden. Wenn die Anwendung an der realen Stelle des angegebenen GPS-Standorts geöffnet wird, wird das in Unity angelegte Objekt am richtigen Ort angezeigt. \autocite{ArGpsLocation}

\subsection{Svelte und Web-Apps} \label{cha:grundlagen:swtech:svelte}

\textbf{Svelte} ist ein modernes JavaScript-Framework zur Erstellung von Benutzeroberflächen. Im Gegensatz zu herkömmlichen Frameworks wie React oder Vue, die den Großteil ihrer Arbeit im Browser erledigen, verlagert Svelte diese Arbeit in die Kompilierungsphase. Das bedeutet, dass der Code während des Build-Prozesses in effizientes, optimiertes JavaScript umgewandelt wird, das direkt im Browser ausgeführt wird. Dadurch wird die Laufzeitlast erheblich reduziert, was zu einer besseren Performance und kürzeren Ladezeiten führt. \autocite{Svelte2024}

Ein wesentlicher Vorteil von Svelte ist seine einfache Syntax, die es Entwicklern ermöglicht, reaktiven Code zu schreiben, ohne auf komplexe Zustandsverwaltungslösungen zurückgreifen zu müssen. Die Komponenten von Svelte bestehen aus HTML, CSS und JavaScript, was die Lernkurve für neue Benutzer verkürzt und die Entwicklungserfahrung vereinfacht. \autocite{Svelte2024}

\textbf{SvelteKit} ist ein Framework zur Erstellung von Svelte-Anwendungen. Es erweitert die Möglichkeiten von Svelte, indem es zusätzliche Werkzeuge und Funktionen bereitstellt, die speziell für die Entwicklung komplexer, leistungsstarker Webanwendungen benötigt werden. SvelteKit vereinfacht die Einrichtung und Strukturierung von Projekten und bietet Funktionen wie Routing, \ac{ssr}, statische Seitengenerierung und eine integrierte Entwicklungsumgebung. \autocite{Svelte2024}

\subsubsection{Vorteile}

Svelte integriert CSS-Styles direkt in die Komponenten mittels \textit{<style>-Tags}. Dadurch entfällt die Notwendigkeit einer zusätzlichen Konfiguration. Alternativ kann auf bestehende CSS-Bibliotheken wie TailwindCSS zurückgegriffen werden. Standardmäßig werden Styles scoped erstellt, um Konflikte zwischen Komponenten zu vermeiden. Dies sorgt für eine sichere und wartbare Codebasis der einzelnen Komponenten durch reduzierte Abhängigkeiten. Svelte verfügt über eingebaute Barrierefreiheitsprüfungen (\textit{Accessibility-Linting}), die auf offensichtliche und auch komplexere Probleme hinweisen. Dies hilft Entwicklern, von Anfang an barrierefreie Anwendungen zu erstellen. \autocite{SvelteExperienceBespoyasov}

Svelte-Pages verfügen standardmäßig über ein einfaches Prerendering, das durch Setzen eines Attributs gesteuert werden kann. SvelteKit bietet auch einfache Einstellungen, um verschiedene Startpfade für das Prerendering zu definieren, was die Flexibilität bei der Konfiguration erhöht. \autocite{SvelteExperienceBespoyasov}

Ein weiteres zentrales Feature von Svelte ist die Reaktivität. Durch die Definition sogenannter \textit{Reactive Statements} kann ein Ablauf beschrieben werden, der ausgeführt wird, wenn eine Abhängigkeit innerhalb dieses Blocks aktualisiert wird. Dies erleichtert das Management von Zustandsänderungen und ermöglicht reaktive und dynamische Benutzerschnittstellen. \autocite{SvelteExperienceBespoyasov}

Die deklarative Natur von Svelte reduziert die Menge des benötigten Codes erheblich. Dies führt zu einer übersichtlicheren Codebasis, welche die Entwicklung und Wartung von Anwendungen vereinfacht.\autocite{SvelteExperienceBespoyasov}

\subsubsection{Nachteile}

Im Vergleich zu bekannteren JavaScript-Frameworks wie React oder Angular ist die Community von Svelte aufgrund der Neuheit des Frameworks noch relativ klein. Dies kann die Fehlersuche erschweren, da einige Bugs möglicherweise nicht so gut dokumentiert oder gelöst sind und somit weniger Ressourcen in Foren wie Stack Overflow zur Verfügung stehen. \autocite{SvelteVsSvelteKit2023}

Ein weiterer Nachteil ist das Fehlen von Entwicklertools oder einer \ac{cli}. Bei Angular können beispielsweise Komponenten oder Services schnell und effizient mit dem Angular \ac{cli} erstellt werden. Diese Art von Tooling fehlt bei Svelte, was den Entwicklungsprozess etwas weniger komfortabel machen kann. \autocite{SvelteVsSvelteKit2023}

Darüber hinaus ist die Verfügbarkeit und Nutzbarkeit der vorhandenen Bibliotheken eingeschränkt. Zwar können diese Bibliotheken problemlos mit einem Paketmanager (z.B. npm von Node.js) installiert werden, jedoch ist es oft mit zusätzlichem Aufwand verbunden, diese in Svelte-Komponenten zu integrieren.\autocite{SvelteVsSvelteKit2023}

\subsubsection{Pages \& Routes}

Eine Svelte-Anwendung ist verzeichnisbasiert. Jedes Verzeichnis entspricht einer Route und kann ein oder mehrere Unterverzeichnisse haben, die die Route erweitern. Jede Route kann eine Seite haben, die immer \textit{+page.svelte} heißt. Svelte verwendet diese Seiten als Hauptanzeige, wenn zu der entsprechenden Route navigiert wird.

\subsubsection{UI Components}

Für die Darstellung, Wiederverwendbarkeit und Konsistenz zwischen verschiedenen Seiten werden Svelte-Komponenten verwendet. Diese bilden die Bausteine für die Darstellung unterschiedlicher Daten. Zusätzlich können UI-Komponentenbibliotheken wie z.B. \textit{flowbite-svelte} eingebunden werden, um standardisierte und ansprechende UI-Elemente wie Buttons, Inputs oder Tabellen zu implementieren. Die Verwendung einer vorgefertigten Bibliothek vereinfacht zudem die Entwicklung und trägt zu einer konsistenten und intuitiven Benutzererfahrung bei.

Jede Svelte-Komponente besteht aus einem Skriptteil und einem Design-/Styleteil. Eine Komponente kann auch andere Komponenten einbinden. In diesem Fall fungiert die übergeordnete Komponente als Elternteil (Parent), während die eingebetteten Komponenten als Kinder (Children) bezeichnet werden.

\subsubsection{Stores}

Um während des Erstellungs- und Bearbeitungsprozesses Daten zu speichern, wird ein Svelte Store verwendet. Dies ermöglicht es verschiedenen Komponenten, auf diese Daten zuzugreifen.

\subsubsection{Node \& Node Modules}

\textit{Node.js} ist eine JavaScript-Laufzeitumgebung, die serverseitige Anwendungen ermöglicht. Sie stellt über den Paketmanager \textit{npm} Werkzeuge zur Verwaltung und Bereitstellung von externen Paket-Abhängigkeiten sowie Entwickler-Tools bereit. Neben dem Betrieb eines Webservers ermöglicht \textit{Node.js} auch die serverseitige Verarbeitung von Api-Anfragen und die Anbindung an Datenbanken.

\subsection{Docker} \label{cha:grundlagen:swtech:docker}

Docker ist eine Technologie zum Deployment von Anwendungen (z.B. Dienste oder Datenbankverbindungen), die zu Testzwecken in der Entwicklungsphase eingesetzt werden können. Docker-Compose und Dockerfiles können ebenfalls zum Deployment von Anwendungen und Systemen verwendet werden.