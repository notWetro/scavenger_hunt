\chapter{Zusammenfassung und Ausblick}
\label{cha:zusammenfassung}
\section{Erreichte Ergebnisse}
\label{sec:ergebnisse}

Im Rahmen der Projektarbeit wurde eine interaktive Schnitzeljagd-Anwendung entwickelt, die speziell auf die Bedürfnisse von Studienanfängern zugeschnitten ist. Das Ziel, den neuen Studierenden den Campus und dessen Einrichtungen auf spielerische Weise näherzubringen, wurde erfolgreich erreicht. Die entwickelte Lösung umfasst ein robustes Backend auf Basis von .NET, welches nicht nur skalierbar ist, sondern auch einfach erweiterbar, um neue Hinweis- und Lösungstypen zu integrieren.

Ein zentraler Aspekt der Implementierung war die Flexibilität des Systems. Die Trennung von Aufgabenstellung und Lösung ermöglicht eine dynamische Erweiterbarkeit, ohne dass erhebliche Änderungen an der bestehenden Infrastruktur erforderlich sind. Sowohl der Web-Editor als auch das Web Game wurden mithilfe von SvelteKit und Komponenten-basiertem Design entwickelt. Dies erlaubt eine leichte Erweiterbarkeit und Anpassung der Benutzeroberflächen und Funktionalitäten. Der benutzerfreundliche Web-Editor ermöglicht die Verwaltung von Schnitzeljagden durch das Anlegen und Editieren von Aufgaben, während das Web Game alle bereits implementierten Hinweis- und Lösungstypen unterstützt, darunter Texte, Bilder, QR-Codes und geolocation-basierte Aufgaben. Dies gewährleistet eine umfassende und interaktive Spielerfahrung.

Insgesamt wurden die gesteckten Ziele erreicht und das System konnte in einem Testlauf erfolgreich eingesetzt werden.

\section{Ausblick}
\label{sec:ausblick}

In Zukunft bietet die entwickelte Lösung zahlreiche Möglichkeiten zur Weiterentwicklung und Anpassung. Eine naheliegende Erweiterung ist die Integration weiterer Hinweis- und Lösungstypen, um die Schnitzeljagden noch abwechslungsreicher zu gestalten. Auch die Einführung von WebAR-Technologien könnte in Betracht gezogen werden, um Augmented Reality-Elemente einzubinden und die Spielerfahrung weiter zu verbessern. Dies wäre insbesondere interessant für Aufgaben, die eine visuelle Interaktion mit der realen Umgebung erfordern.

Ein weiterer Entwicklungsstrang könnte die Implementierung von Gamification-Elementen sein, wie etwa Ranglisten oder Belohnungssysteme, um die Motivation der Teilnehmer zu steigern. Darüber hinaus könnten Sicherheitsaspekte, wie die Verwendung kryptographischer Hashfunktionen zur Speicherung von Benutzerdaten, weiter ausgearbeitet werden, um die Datenintegrität und den Datenschutz zu verbessern.