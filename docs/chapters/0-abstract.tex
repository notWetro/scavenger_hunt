Die vorliegende Projektarbeit dokumentiert umfassend das Projekt \textit{Web-basierte Schnitzeljagd}. Das Projekt umfasst den Entwurf, die Implementierung und Inbetriebnahme eines Schnitzeljagd-Editors, der zur Erstellung und Bearbeitung bestehender Schnitzeljagden dient. Zudem wurde eine Web-Anwendung entwickelt, die es den Teilnehmern ermöglicht, an Schnitzeljagden teilzunehmen und diese zu absolvieren.

Ursprünglich war geplant, eine mobile Anwendung mit AR-Unterstützung auf Basis von Unity zu entwickeln. Im Laufe des Projekts wurde jedoch entschieden, stattdessen auf eine Web-basierte Lösung umzusteigen. Die Gründe für diese Entscheidung und die damit verbundenen Überlegungen werden ausführlich in der Arbeit dokumentiert. Trotz dieses Wechsels wurde intensiv mit AR-Technologien, insbesondere WebAR, gearbeitet, um die Möglichkeiten der Integration von AR-Elementen in eine Webumgebung zu erforschen.

Die Arbeit bietet einen umfassenden Einblick in die technischen Details der Projektrealisierung, beginnend mit einer Einführung in relevante Frameworks, Technologien und Bibliotheken, über die Konzeption und den Entwurf, bis hin zur konkreten Implementierung und den Herausforderungen, die während des Projektdurchlaufs auftraten.

Im Rahmen der Projektarbeit wurden vielfältige Aspekte der Software-Entwicklung behandelt, wobei der Schwerpunkt auf der Entwicklung einer skalierbaren und flexiblen Weblösung lag. Besondere Aufmerksamkeit wurde der Verwendung von Microservices, der Interprozesskommunikation über REST-Apis und der Integration von WebAR-Technologien gewidmet.
