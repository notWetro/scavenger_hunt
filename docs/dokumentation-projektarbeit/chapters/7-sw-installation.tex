\chapter{Inbetriebnahme}
\label{cha:inbetriebnahme}

Aufgabe des Kapitels Inbetriebnahme ist es, die Überführung der in 
Kapitel \ref{cha:implementierung} entwickelte Lösung in das betriebliche 
Umfeld aufzuzeigen. Dabei wird beispielsweise die Inbetriebnahme eines 
Programms beschrieben oder die Integration eines erstellten 
Programmodules dargestellt.

Bei der Software-Erstellung entspricht dieses Kapitel der 
Auslieferungsphase (Deployment) im \ac{rup}.

%\chapter{Evaluierung}
%Aufgabe des Kapitels Evaluierung ist es, in wie weit die Ziele der Arbeit erreicht wurden. Es sollen also die erreichten Arbeitsergebnisse mit den Zielen verglichen werden. Ergebnis der Evaluierung kann auch sein, das bestimmte Ziele nicht erreicht werden konnten, wobei die Ursachen hierfür auch außerhalb des Verantwortungsbereichs des Praktikanten liegen können.