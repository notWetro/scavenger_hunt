\chapter{Grundlagen}
\label{cha:grundlagen}

In diesem Kapitel das für das Praktikum relevante Grundlagenwissen 
dargestellt. Der Praktikant soll hierzu das ihm durch Vorlesungen 
bekannte, bzw. durch Recherchen vertiefte theoretische Wissen 
darstellen, das für die Lösung der im Praktikum gestellten Probleme 
notwendig ist.

Dabei ist darauf zu achten, nur solche Inhalte in das Grundlagenkapitel 
aufzunehmen, die später auch verwendet werden (Problembezogenheit). 
Ebenso ist auf eine ausreichend tiefe und vollständige Darstellung der 
Grundlagen zu achten.

Für die Erstellung des Literaturverzeichnisses 
wird das Werkzeug JabRef\autocite{JabRef:JabRef} verwendet. 

Sie können aber auch das Werkzeug Citavi\autocite{SAS:Citavi} benutzen
und dort nach \textsc{Bib}\TeX{} exportieren.

\section{Grundlagengebiet A}
\label{sec:grundlagengebieta}

\subsection{Definition AA}
\label{sub:definitionaa}

\subsection{Definition AB}
\label{sub:definitionab}

\section{Grundlagengebiet B}
\label{sec:grundlagengebietb}

\subsection{Definition BA}
\label{sub:definitionBa}

\subsection{Definition BB}
\label{sub:definitionbb}