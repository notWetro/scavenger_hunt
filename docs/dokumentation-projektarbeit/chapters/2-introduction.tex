\chapter{Einleitung} \label{cha:einleitung}

\section{Motivation}

Die ersten Tage und Wochen an der Hochschule können für neue Studierende eine herausfordernde und überwältigende Zeit sein. Sie müssen sich in einer neuen Umgebung zurechtfinden, viele neue Menschen kennenlernen und gleichzeitig den akademischen Anforderungen gerecht werden. Eine effektive Methode, um den Einstieg zu erleichtern, ist eine Einführungstour, die jedoch oft nur begrenzt interaktiv und ansprechend ist.

Traditionelle Einführungstouren sind meist sehr statisch und linear. Studierende folgen einer festgelegten Route und hören passiv zu, während Informationen vermittelt werden. Dies führt oft dazu, dass wichtige Details nicht in Erinnerung bleiben, da die Interaktion und das eigenständige Entdecken fehlt.

Zudem bieten solche Touren selten die Möglichkeit zur aktiven Teilnahme und Mitgestaltung. Die Studierenden sind Zuschauer statt Akteure, was die Aufnahmefähigkeit und das Engagement reduziert. Ohne die Möglichkeit, selbst Entscheidungen zu treffen oder Aufgaben zu lösen, bleibt der Lerneffekt gering und die Tour wird schnell langweilig.

Schließlich ist die Interaktion zwischen den neuen Studierenden bei herkömmlichen Einführungstouren meist eingeschränkt. Da der Fokus auf der Vermittlung von Informationen liegt, bleibt wenig Raum für soziale Interaktionen und Teamarbeit. Somit fällt es den Studierenden schwer, frühzeitig Kontakte zu knüpfen, um eine Lerngruppe zu finden und ein Gemeinschaftsgefühl entwickeln zu können.

Diese Herausforderungen verdeutlichen die Notwendigkeit für innovativere und interaktivere Ansätze, wie sie durch eine Schnitzeljagd geboten werden können.

\section{Zielsetzung}

Das Ziel dieses Projekts ist die Entwicklung eines Systems zur Erstellung, Anmeldung und Durchführung von Schnitzeljagden an der Hochschule Aalen. Durch eine Schnitzeljagd werden die Studierenden nicht nur spielerisch mit den verschiedenen Gebäuden, Räumen und Einrichtungen vertraut gemacht, sondern auch zur aktiven Teilnahme und Zusammenarbeit angeregt. Dies fördert nicht nur das Verständnis der Campusstruktur, sondern auch das Gemeinschaftsgefühl und den Zusammenhalt unter den neuen Studierenden.

\section{Vorgehen}

Um das beschriebene Projekt zu realisieren, wurden folgende Projekt-Phasen durchlaufen:

\subsubsection{Technologische Forschungsphase}

Die Durchführung einer Schnitzeljagd sollte über das Smartphone erfolgen. Im Verlauf der Jahre haben sich einige Möglichkeiten, Software für mobile Geräte zu entwickeln, durchgesetzt, die jeweils ihre Vor- und Nachteile besitzen. In dieser Projektphase war es wichtig, die vielen unterschiedlichen Technologien zu erforschen und eine für das Projekt passende Plattform zu wählen, in welcher die Durchführung der Schnitzeljagden erfolgen kann.

\subsubsection{Entwurfsphase}

Nachdem das grundlegende, projektrelevante technologische Wissen verfeinert worden war, wurde ein erster Entwurf der fundamentalen Architektur vorgestellt. Nachdem dieser ausreichend ausgereift war, konnten erste Workflows der Benutzer erstellt werden. Sobald die Baseline für die Implementierung festgelegt worden war, konnten erste Aufgaben verteilt werden.

\subsubsection{Implementierungsphase}

Die einzelnen Funktionalitäten wurden daraufhin implementiert und getestet. Hierbei war es entscheidend, die geplanten Features schrittweise zu realisieren und kontinuierlich mit den Anforderungen zu überprüfen.
