\chapter{Einleitung}
\label{cha:einleitung}

Die Einleitung dient dazu, beim Leser Interesse für die Inhalte 
Praxissemesterberichts zu wecken, die behandelten Probleme aufzuzeigen 
und die zu ihrer Lösung entwickelten Konzepte zu beschreiben.

\section{Motivation}
\label{sec:motivation}

In der Motivation wird dargestellt, welche Bedeutung die im 
Praxissemester zu entwickelnden Lösungen für das betreuende Unternehmen 
haben. Es wird beispielsweise aufzeigt, in welches Produkt sie eingehen, 
welcher Ablauf verbessert werden soll etc.

\section{Problemstellung und -abgrenzung}
\label{sec:problemstellung}

Die Problemstellung dient dazu, das zu lösende Problem klar zu 
definieren und abzugrenzen. Der Praktikant soll ein klares Verständnis 
des zu lösenden Problems haben. Insbesondere soll auch verhindert 
werden, dass zu viele Probleme gleichzeitig angegangen werden. Eine 
Negativabgrenzung verhindert, dass beim Leser später nicht erfüllte 
Erwartungen geweckt werden.

\section{Ziel der Arbeit}
\label{sec:ziel}

Mit dem Ziel der Arbeit wird der angestrebte Lösungsumfang festgelegt. An diesem Ziel wird entschieden, ob das Praktikum erfolgreich absolviert wurde oder nicht.

\section{Vorgehen}
\label{sec:vorgehen}

Nachdem mit Problemstellung und Ziel gewissermaßen Anfangs- und Endpunkt 
des Praktikums beschrieben sind, wird hier der zur Erreichung des Ziels 
eingeschlagene Weg vorgestellt. Dazu werden typischerweise die folgenden 
Kapitel und ihr Beitrag zur Erreichung des Ziels der Arbeit kurz 
beschrieben. Die folgenden Kapitel sind ein – möglicher – Aufbau, 
Abweichungen können durchaus notwendig sein. Zur Darstellung des 
Vorgehens ist eine grafische Darstellung sinnvoll, bei der die einzelnen 
Lösungsschritte und ihr Zusammenhang dargestellt werden. Ein Beispiel 
hierfür findet sich in Abbildung.

%\begin{figure}[htbp]
%  \centering
%  \includegraphics[height=0.9\textheight]{images/ausarbeitung.jpg}
%  \caption{vorgehen nach \autocite{Schmidt:Geschaeftsprozesse}}
%  \label{fig:1}
%\end{figure}